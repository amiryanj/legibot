\section{Dynamic Legibility}

\begin{itemize}
    \item
    RQ1: How can legible motions be designed and implemented in HRC systems to enhance the transparency and understand robot actions and intentions?
%    \item
%    RQ2: What methods and technologies can be employed to enable the robot to take the user's perspective, and how does this enhance the efficiency and naturalness of collaborative interactions?
%    \item
%    RQ3: How can intention prediction in HRC be improved, and what are the key factors that influence the accuracy and speed of predicting user intentions?
\end{itemize}

Three assumptions for our problem:
\begin{itemize}
    \item
    We are developing a handover task between a robot and a human.
    \item
    The robot will track the human's head pose and gaze direction to compute the legibility of its motion.
    \item
    The generation of legible motions is done in real-time, and the robot should be able to generate legible motions for different scenarios.
\end{itemize}


Trajectory synthesis within dynamic environment:

\subsection{Background: Legibility}

\begin{equation}
    \mathbf{L}(\xi) = P(G^* |\xi_{S \rightarrow Q}) =
    \frac{\exp (-C(\xi_{S \rightarrow Q})) \int \exp( -C(\xi_{Q \rightarrow G}))}
         {\int \exp (-C(\xi_{S \rightarrow G}))}
\end{equation}

\subsection{Projection of the trajectory for a given observer}
Replace $C(\xi)$ with $T(C(\xi))$ where $T$ is the projection function from the observer's frame.

\subsection{Challenges}

\begin{itemize}
    \item \textbf{Moving Observer:} when the observer position moves from point $O_A$ to $O_B$, the robot's motion should adapt.
    should we re-calculate \ref{eq:legibility} from scratch? or can we use the previous calculation?\\
    \textit{Answer:} Now we should consider $T$ as a function of time, say $T^t$:

    \item \textbf{Prediction:} when we generate trajectory for time $t_0+dt$ in the future we need to have a prediction of the observer's position at that time.\\
    \textit{Answer:} It can be simply trajectory extrapolation, or we can also predict the intention of the user and use that to predict the observer's position.
    The prediction can be represented as a probability distribution (e.g. a Gaussian distribution) over the observer's position.
    Which means, the projected cost of $C(\xi_{Q \rightarrow G})$ will be a distribution over the cost ...
\end{itemize}

\subsection{Scenario:}
Here, we explain the scenario that we are working on ...

