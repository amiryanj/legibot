%! Author = javad
%! Date = 04/01/2024

\section{Synthesis}

The main motivation of this part is to propose a practical and simple algorithm to generate legible motions for mobile robots.
And for this we need to highlight that the synthesis problem has not received as much attention as the evaluation in the literature.

Let's start with the legibility function, proposed by Dragan et al. \cite{dragan2013legibility}. 
In fact, optimizing the legibility function directly is not always possible [REF].
This is because that $\Delta \mathbf{L}(\xi) = 0$ or $\mathbf{L}(\xi) = 1$ might not have a solution in the finite space.
For this reason, we might need to add some constraints to the equation to make it solvable.
In the original work, Dragana et al. [X] this is done by adding a regularizer that discourage increasing the path length.
$L(\xi) = legibility(\xi) - \lambda C(\xi)$
where $C(\xi)$ is the path length, and $\lambda$ is a constant.
%    Also, Dragan and Srinivasa [X], introduce a trust region constraint on the optimization
%     to ensure the motion does not become too surprising or unpredictable to the observer.
%    This approach in the end, turns to finding a good value for a parameter $\beta$, using a user study.
Moreover, due to the iterative nature of this approach, it can compromise the real-time performance of the system.
%    Other works [NIKOLAIDIS] a gradient ascent optimization is called N times, and the solution of the last iteration is returned.


In this work, we propose to use local planning to generate legible motions.
Back to local planning algorithms, such as Dynamic Window Approach (DWA) [REF], are designed to generate a motion that is feasible and optimal in terms of a cost function.
%
In this family of algorithms, instead of finding a path from the start to the goal,
the robot is trying to find a motion that is optimal in terms of a cost function, over a short horizon, or a Window.
This window, denoted by $w$, is defined in such a way that the robot can stop before hitting an obstacle,
by incorporating the robot's kinematic and dynamic constraints.
But also, to ensure the real-time performance of the algorithm, in environments where a full plan is too expensive to compute.
For example in a manipulation task, where we have access to a sufficient processing power,
we can assume that the robot can compute a full plan to the goal.

Let $\xi_{t_0:t_w}$ be the path we want to plan, from the current time $t_0$ to the time $t_w$, then we can define a cost function $C(\xi_{t_0:t_w})$ to optimize over the window $w$.
This cost function can be in the form of a weighted sum of different terms, such as the distance to the goal, the distance to the obstacles, and the smoothness of the path:

\begin{equation}
    \label{eq:cost_general}
    C(\xi_{t_0:t_w}) = \sum_i \alpha_i C_{task}^i(\xi_{t_0:t_w})
\end{equation}

\noindent
where $C_{task}^i(.)$ is the cost of the $i$-th term, and $\alpha_i$ is the weight of the $i$-th term.

We propose a new term to this cost function, to incorporate the legibility of the motion,
and we call it $C_{I}^j(\xi_{t_0:t_w})$, representing the illegibility of the motion for the $j$-th observer.

%\begin{equation}
%    \label{eq:cost_leg}
%    C_{I}^j(\xi_{t_0:t_w}) = \sum_{k=1}^{N_j} \int_{t=t_0}^{t_w} \mathbf{L}(\xi_t, \mathbf{O}_k^j) dt
%\end{equation}

Let's assume that the sub-trajectory $\xi_{t:t+dt}$ incurs an extra cost of $\Delta C_{I}^j(\xi_{t:t+dt})$ for the $j$-th observer.
%Then, we can define the cost of the path as:

We assume that this cost is proportional to the probability of the observer not being able to predict the true intention of the robot, i.e.:

\begin{equation}
    \label{eq:cost_leg}
%    \Delta C_{I}^j(\xi_{t:t+dt}) \propto \frac{dt}{\exp \left( p^j(G^* | \xi_{t_0:t+dt}) \right)}
    \Delta C_{I}^j(\xi_{t:t+dt}) \propto dt \times  p^j(g \neq G^* | \xi_{t_0:t+dt})
\end{equation}

\noindent
Here, instead of using the approximation used in Eq.~(\ref{eq_dragan9}),
which assumes an observer will compare the cost of the current path to the cost of the optimal path,
we only assume the impact of the last action on the observer's belief.

\begin{equation}
    \label{eq:cost_leg_approx}
    \Delta C_{I}^j(\xi_{t:t+dt}) \propto dt \times  p^j(g \neq G^* | \xi_{t:t+dt})
\end{equation}

This approximation helps us to, first, avoid the need for computing the cost of the global optimal path,
and second, to make the cost function more computationally efficient.

\noindent

\begin{equation}
    \label{eq:prob_leg}
    p^j(g \neq G^* | \xi_{t:t+dt}) \propto \sum_{g \neq G^*} L^j(g | \xi_{t:t+dt})
\end{equation}

We can approximate $L^j(g | \xi_{t:t+dt})$ with the projected cosine distance between $\xi_{t:t+dt}$ and the optimal path from $x_t$ to $g$:

\begin{equation}
    \label{eq:prob_leg_approx}
    p^j(g \neq \mathrm G^* | \xi_{t:t+dt}) \approx \sum_{g \neq \mathrm G^*}  d (\measuredangle \mathrm T^j \times \mathbf{v}_{t:t+dt}(.|\mathrm G^*),
                                                                 \measuredangle \mathrm T^j \times \mathbf{v}^*_{t:t+dt}(.|g))
\end{equation}

\noindent
where $\mathrm T^j$ is the transformation matrix from the world frame to the observer's frame $j$,
and $\mathbf{v}_{t:t+dt}(.|\mathrm G^*)$ is the velocity vector of the robot at time $t$,
Here we compute the optimal paths also by solving Eq. (\ref{eq:cost_general}) for each goal $g$, and

%\quote{\hl{Q: Can this be seen as a Dynamic Programming approach to the problem?! Or is it just a greedy approach?}}
