%! Author = javad
%! Date = 04/01/2024

\section{Synthesis}

% ================ Synthesis vs. Evaluation ================
The main motivation of this part is to propose a practical and simple algorithm to generate legible motions for mobile robots.
First of all, synthesizing legible behaviors can be explained as ``finding behaviors that convince one or multiple observer that the robot is doing what it is supposed to do''.
This is beyond the evaluation problem, where we are trying to find a metric to evaluate the legibility of a given behavior \cite{dragan2013generating}.
And, we would like to highlight that the synthesis problem has not received as much attention as the evaluation in the literature,
mainly because of the complexity of the problem.
%



\vspace{0.3cm}
% ================ Setup the problem ================
We assume our mobile robot has a goal of reaching a target $\mathrm{G}^*$ that might move over time.
However, there are other goals $\mathrm{G}^i$ that are irrelevant to the robot's task, all forming a set of goals $\mathcal{G}$.
In the environment, there will be a set of observers $\mathcal{O}$, that contains a positive number of observers $O^j$ that are interested in the robot's behavior.
The robot is supposed to reach the goal $\mathrm{G}^*$ in a way that is legible for all the observers $O^j$.
Note that, the observers set $\mathcal{O}$ and the goals set $\mathcal{G}$ might be the same, but not necessarily.
For example, in a restaurant scenario, the robot might want to deliver a dish to a customer ($\mathrm{G}^*$),
while there might be other customers in the restaurant $\mathcal{G} \neq \mathrm{G}^*$,
and we assume that all of them are interested in the robot's behavior ($\mathcal{O} = \mathcal{G}$).
%
Finally, the robot is moving in an environment $\mathcal{E}$, with an obstacle-free space $\mathcal{E}_{free}^t$ that might also change over time.
We are interested in finding a path $\xi_{t_0:t_w}$ from the current time $t_0$ to the time $t_w$ that is legible for all the observers $O^j$,
while satisfying the robot's task, i.e. reaching the goal $\mathrm{G}^*$, avoiding the static and dynamic obstacles,
moving within the robot's kinematic and dynamic constraints, and so on.


% ================ Local Planning ================
\vsapce{~}

In this work, we propose to use local planning to generate legible motions.
Back to local planning algorithms, such as Dynamic Window Approach (DWA) \cite{DWA1997}, are designed to generate a motion that is feasible and optimal in terms of a cost function.
%
In this family of algorithms, instead of finding a path from the start to the goal,
the robot is trying to find a motion that is optimal in terms of a cost function, over a short horizon, or a Window.
This window, denoted by $w$, is defined in such a way that the robot can stop before hitting an obstacle,
by incorporating the robot's kinematic and dynamic constraints.
But also, to ensure the real-time performance of the algorithm, in environments where a full plan is too expensive to compute.
%%%% Extra
%For example, in a manipulation task, with short-term goals and with access to a sufficient processing power,
%we can assume that the robot can compute a full plan to the goal.

Let $\xi_{t_0:t_0+w}$ be the path we want to plan, from the current time $t_0$ over a window $w$.
We define a cost function
%    $C(\xi_{t_0:t_0+w})$
$C(\xi)$
to optimize over this window.
This cost function can be in the form of a weighted sum of different terms, such as the distance to the goal, the distance to the obstacles, and the smoothness of the path:

\begin{equation}
    \label{eq:cost_general}
    C(\xi_{t_0:t_0+w}) = \sum_i \alpha_i C_{task}^i(\xi_{t_0:t_0+w})
\end{equation}

\noindent
where $C_{task}^i(.)$ is the cost of the $i$-th term, and $\alpha_i$ is the weight of the $i$-th term.
%
We propose an extra term to this cost function, to incorporate the legibility of the motion,
and we call it $C_{I}^j(\xi_{t_0:t_0+w})$, representing the illegibility of the motion for the $j$-th observer.

%\begin{equation}
%    \label{eq:cost_leg}
%    C_{I}^j(\xi_{t_0:t_w}) = \sum_{k=1}^{N_j} \int_{t=t_0}^{t_w} \mathbf{L}(\xi_t, \mathbf{O}_k^j) dt
%\end{equation}

We assume the trajectory $\xi_{t_0:t_0+w}$ is a sequence of sub-trajectories $\xi_{t:t+dt}$, or simply velocity vector $\mathbf{v}_t$, where $dt$ is the time step.
Hence, a sub-trajectory $\xi_{t:t+dt}$ incurs an extra cost of $\Delta C_{I}^j(\xi_{t:t+dt})$ for the $j$-th observer.
We assume that this cost is proportional to the probability of the observer not being able to predict the true intention of the robot, i.e.:

\begin{equation}
    \label{eq:cost_leg}
%    \Delta C_{I}^j(\xi_{t:t+dt}) \propto \frac{dt}{\exp \left( p^j(G^* | \xi_{t_0:t+dt}) \right)}
    \Delta C_{I}^j(\xi_{t:t+dt}) \propto  \mathrm{P}^j(G \neq G^* | \mathbf{v}_t) ~dt
\end{equation}

\noindent
Here, instead of assuming the observer has access to the full path $\xi_{t_0:t+dt}$,
we assume the observer only has access to the current state of the robot $\xi_{t:t+dt}$.
%which assumes an observer will compare the cost of the current path to the cost of the optimal path,
%we only assume the impact of the last action on the observer's belief.
%
%\begin{equation}
%    \label{eq:cost_leg_approx}
%    \Delta C_{I}^j(\xi_{t:t+dt}) \propto dt \times  L^j(g \neq G^* | \xi_{t:t+dt})
%\end{equation}
%
This helps us to, first, to make the cost function more computationally efficient.
and second, to make it more realistic, as we have no information if the robot has been in the observer's field of view in the past,
or if the observer has actively been paying attention to the robot's behavior.

\noindent

\begin{equation}
    \label{eq:prob_leg}
    \mathrm{P}^j(G \neq G^* | \mathbf{v}_t) = \sum_{G \neq G^*} \mathrm{P}^j(G | \mathbf{v}_t)
\end{equation}

\noindent
We can approximate $\mathrm{P}^j(G | \mathbf{v}_t)$ with the projected cosine distance between $\mathbf{v}_t$ and the optimal path from $\mathbf{x}_t$ to $G \neq G^*$:

\begin{equation}
    \label{eq:prob_leg_approx}
    \mathrm{P}^j(G \neq \mathrm G^* | \mathbf{v}_t) \approx
                        \sum_{G \neq \mathrm G^*}  d \measuredangle(
                                                                        \mathrm T^j \mathbf{v}_t,
                                                                        \mathrm T^j \mathbf{v}^*_t(.|G))
\end{equation}

\noindent
where $\mathrm T^j$ is the transformation matrix from the world frame to the observer's frame $j$,
and $\mathbf{v}_{t:t+dt}(.|\mathrm G^*)$ is the velocity vector of the robot at time $t$,
Here we compute the optimal paths also by solving Eq. (\ref{eq:cost_general}) for each goal $G$,
and then we compute the cosine distance between the velocity vector of the robot and the optimal velocity vector for each goal $G$.

%\quote{\hl{Q: Can this be seen as a Dynamic Programming approach to the problem?! Or is it just a greedy approach?}}

